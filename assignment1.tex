\chapter{Assignment \#1: Basic Rankine Cycles}
\label{ch:ass1}

Use MATLAB and EasyProp to answer the following questions
\begin{fullwidth}
\begin{enumerate}
\item Steam flowing at 15.9 lb$_{\text{m}}$/s enters the turbine of a simple Rankine cycle power plant at 1000 psia, 800$^{\circ}$F and exits at 2 psia.  Saturated liquid exits the condenser.  Assume that the turbine and pump are isentropic. (\textbf{Note:} 1 kW = 3412 BTU/hr, 1 hp = 2545 BTU/hr)
\begin{enumerate}
\item How many degrees superheat (degrees above saturation temperature) is the steam exiting the boiler?
\item What is the quality of the steam exiting the turbine?
\item Determine the power output of the turbine in kW and hp.
\item Determine the power input to the pump in hp.
\item Determine the cycle thermal efficiency.
\end{enumerate}

\vspace{1.0cm}

\item In a 2-MW (turbine power output) Rankine cycle, saturated vapor leaves the steam generator at 2 MPa and expands in the turbine to an outlet condition of 15 kPa, 94\% quality.  Saturated liquid leaves the condenser.  Assume the pump is isentropic.
\begin{enumerate}
\item What is the isentropic efficiency of the turbine?
\item Determine the flow rate fo the steam [kg/s].
\item Determine the cycle thermal efficiency.
\end{enumerate}

\vspace{1.0cm}
\item Investigate the effect of the condenser pressure on the performance of a simple ideal Rankine cycle.  Turbine inlet conditions of the steam are maintained constant at 5 MPa and 500$^{\circ}$C while the condenser pressure is varied from 5 to 100 kPa.  Determine the thermal efficiency of the cycle and plot it against the condenser pressure.  Write a short essay (no need to exceed 500 wrods) discussing the results.  Be sure to include a discussion on the extent to which the designer of an energy conversion system based on the Rankine Cycle has control of condenser pressure.

\end{enumerate}
\end{fullwidth}
