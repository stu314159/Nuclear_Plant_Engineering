\chapter{EasyProp Interface Description}
\label{ch:ep_interface}

\section{Introduction}
This appendix is provided as a overview of the most commonly used functions in the EasyProp interface.

In each section below, we will assume that the user has initialized an object in the MATLAB environment from which to obtain properties from EasyProp.  As an example, the listing below initializes an object to obtain water properties in SI units.

To run this code, create a MATLAB script containing the code in the listing below.  You should have a copy of the library EasyProp.py in the same folder.

\begin{fullwidth}
\begin{lstlisting}
%% Prepare the Environment
clear
clc
close 'all'

EasyProp_path=' '; % <-- enter relative or absolute path to folder containing EasyProp.py
if count(py.sys.path,EasyProp_path) == 0  % <-- see if desired directory is on path
    insert(py.sys.path,int32(0),EasyProp_path); %<-- if not; add it.
end

%% Initialize the fluid:
fluid = py.EasyProp.simpleFluid('Water','SI');
\end{lstlisting}
\end{fullwidth}

\section{Thermodynamic Properties of Simple Fluids}
\subsection{Enthalpy}

\begin{fullwidth}
\begin{lstlisting}
%% Enthalpy - SI units: kJ/kg   USCS units: BTU/lbm
h1 = fluid.hL_p(101.3); % saturated liquid enthalpy at 101.3 kPa
h2 = fluid.hV_p(8000.0); % saturated vapor enthalpy at 8000 kPa
h3 = fluid.h_ps(3500,1.25); % enthalpy at 3500 kPa, 1.25 kJ/kg-K
h4 = fluid.h_pT(101.3,200); % enthalpy at 101.3 kPa, 200 C
h5 = fluid.h_Tx(350,0.5); % enthalpy at 350 C, 50% quality
h6 = fluid.h_px(8000,0.25); % enthalpy at 8000 kPa, 25% quality
% also: h_pv(P,V)
\end{lstlisting}
\end{fullwidth}

\subsection{Entropy}
\begin{fullwidth}
\begin{lstlisting}
%% Entropy - SI units: kJ/kg-K  USCS units: BTU/lbm-R
s1 = fluid.sL_p(101.3); % saturated liquid entropy at 101.3 kPa
s2 = fluid.sV_p(8000.0); % saturated vapor entropy at 8000 kPa 
s3 = fluid.s_Tx(200,0.4); % saturated mixture entropy at 200 C, x=0.4
s4 = fluid.s_pT(101.3,200); % entropy at 101.3 kPa, 2000 C
s5 = fluid.s_ph(300.0,1234); % entropy at 300 kPa, h=1234 kJ/kg
s6 = fluid.s_px(300, 0.5); % entropy at 300 kPa, 50% quality
% also s_pu(P,U)
\end{lstlisting}
\end{fullwidth}

\subsection{Temperature and Pressure}
\begin{fullwidth}
\begin{lstlisting}
%% Temperature - SI units: C  USCS units: F
T1 = fluid.T_ps(101.3,0.5); % temperature at 101.3 kPa, 0.5 kJ/kg-K
T2 = fluid.T_ph(101.3,1234); % temperature at 101.3 kPa, 1234 kJ/kg
T3 = fluid.T_pv(200.0,0.002); % temperature at 200 kPa, 0.002 m^3/kg
T4 = fluid.Tsat_p(8000); % saturation temperature at 8000 kPa
T5 = fluid.T_crit(); % critical temperature for the fluid (C for SI, F for USCS)
T6 = fluid.T_pu(101.3,200.); % temperature at 101.3 kPa, 200 kJ/kg 
T7 = fluid.T_sv(1.5,3.5); % temperature at 1.5 kJ/kg-K and 3.5 m^3/kg
%% Pressure - SI units: kPa USCS units: psia

P1 = fluid.Psat_T(300); % saturation pressure at 300 C 
P2 = fluid.P_vT(0.002,300.); % pressure at 0.002 m^3/kg, 300 C
P3 = fluid.P_crit(); % critical pressure for the fluid (kPa or psia)
% also: P_sT(S,T)
%       P_sh(S,H)

\end{lstlisting}
\end{fullwidth}

\subsection{Internal Energy}
\begin{fullwidth}
\begin{lstlisting}
U1 = fluid.uL_p(101.3); % internal energy of saturated liquid at 101.3 kPa
U2 = fluid.uV_p(101.3); % internal energy of saturated vapor at 101.3 kPa
U3 = fluid.u_px(101.3,0.5); % internal energy of saturated mixture at 101.3 kPa, 50% quality
U4 = fluid.u_Tx(200,0.5); % internal energy of saturated mixture at 200 C, 50 % quality
U5 = fluid.u_pT(101.3,200); % internal energy of a fluid at 101.3 kPa, 200 C
U6 = fluid.u_ps(101.3,1.5); % internal energy of a fluid at 101.3 kPa, 1.5 kJ/kg-K
U7 = fluid.u_Tv(20,0.002); % internal energy of a fluid at 20 C, 0.002 m^3/kg
\end{lstlisting}
\end{fullwidth}

\subsection{Specific Volume and Quality}
\begin{fullwidth}
\begin{lstlisting}
%% specific volume - SI units: m^3/kg, USCS units: ft^3/lbm
V1 = fluid.v_ph(101.3,2300.); % specific volume of fluid at 101.3 kPa, 2300.0 kJ/kg
V2 = fluid.v_pu(101.3,500.); % specific volume of fluid at 101.3 kPa, 500. kJ/kg
V3 = fluid.v_Tx(200.,0.5); % specific volume for saturated mixture at 200 C, 50% quality
V4 = fluid.v_pT(101.3,200.); % specific volume for 101.3 kPa, 200 C

%% Quality
x1 = fluid.x_ph(100,700); % quality at 100 kPa, 700 kJ/kg enthalpy
x2 = fluid.x_pu(101,500); % quality at 101 kPa, 500 kJ/kg internal energy

\end{lstlisting}
\end{fullwidth}

\subsection{Viscosity, Thermal Conductivity, and Prandtl Number}
\begin{fullwidth}
\begin{lstlisting}
%% viscosity - SI units: Pa-s (kg/m-s)  USCS units: lbm/ft-s
mu1 = fluid.mu_pT(101.3,200); % viscoisty at 101.3 kPa, 200 C

%% thermal conductivity - SI units: kW/m-K USCS units: BTU/ft-sec-R
k1 = fluid.k_pT(101.3,200); % thermal conductivity at 101.3 kPa, 200 C

%% Prandtl Number 
Pr1 = fluid.Prandtl_pT(101.3,200); % fluid Prandtl number at 101.3 kPa, 200 C
\end{lstlisting}
\end{fullwidth}

\subsection{Specific Heat and Ideal Gas Constants}
\begin{fullwidth}
\begin{lstlisting}
helium = py.EasyProp.simpleFluid('He','USCS');
%% Specific heats - SI units kJ/kg-K, USCS units: BTU/lbm-R
Cp = helium.Cp_pT(14.7,200); % specific heat at constant pressure for Helium at 14.7 psia, 200 F
Cv = helium.Cv_pT(14.7,200); % specific heat at constant volume
gamma = helium.gamma_pT(14.7,200); % ratio of: Cp/Cv for Helium

%% Gas Constant - SI units kJ/kg-K  USCS units: BTU/lbm-R
R1 = helium.R_pT(14.7,200); % Gas constant for Helium at 14.7 psia, 200 F 

%% Molecular weight - SI kG/kmol USCS units: lbm/mol
M = helium.M(); 

\end{lstlisting}
\end{fullwidth}

\section{Humid Air Object and Functions}

\begin{fullwidth}
\begin{lstlisting}
%% Humid Air Object and functions
humidAir = py.EasyProp.HumidAir('SI'); % option: 'SI' | 'USCS'

% humidity ratio as a function of temp, pressure, and relative humidity
w = humidAir.w_PTR(101.3,20,0.2); % no units

% enthalpy as a function of temp, pressure, and relative humidity
h = humidAir.h_PTR(101.3,20,0.2);% kJ/kg or BTU/lbm

\end{lstlisting}
\end{fullwidth}



