\chapter{Example - Recompressing Regenerating S-CO$_{2}$ Analysis}
\label{app:rrsco2_ex}
\index{Brayton cycle, recompressing regenerating S-CO$_{2}$}

To run this code, create a MATLAB script containing the code in the listing below.  You should have a copy of the library EasyProp.py in the same folder.

\begin{fullwidth}
\begin{lstlisting}
%% SCO-2 energy conversion example
clear
clc
close 'all'

%% Add current directory to the Python Path
EasyProp_path = ' '; 
if count(py.sys.path,EasyProp_path) == 0  
    insert(py.sys.path,int32(0),EasyProp_path); 
end

%% Initialize Fluid Property object
fluid = 'CO2';
units = 'SI';
gas = py.EasyProp.EasyProp(fluid,units);

To_C = 25; % C
To = To_C + 273; % K
Po = 101.3; % kPa
ho = gas.h_pT(Po,To_C);
so = gas.s_pT(Po,To_C);


% function for specific flow exergy
ef_fun = @(h_val,s_val) h_val - ho - To*(s_val - so);

%% Initialize State Point Data Arrays

numSP = 8;
h = nan(numSP,1);
h_s = nan(numSP,1);
s = nan(numSP,1);
s_s = nan(numSP,1);
T = nan(numSP,1);
P = nan(numSP,1);
x = nan(numSP,1);
ef = nan(numSP,1);

eta_turb = 1;
eta_c = 1;
eta_rc = eta_c;

xi_ltr = 0.90;
xi_htr = 0.90;

Pmax = 20000; % kPa
Pmin = 7700; % kPa

%% state point property calculations
P(1) = Pmin; % kPa
T(1) = 32; % C
h(1) = gas.h_pT(P(1),T(1));
s(1) = gas.s_pT(P(1),T(1));
ef(1) = ef_fun(h(1),s(1));

% 1 -> 2 compression
P(2) = Pmax; % kPa
s_s(2) = s(1);
h_s(2) = gas.h_ps(P(2),s_s(2));
h(2) = h(1) - (h(1) - h_s(2))/eta_c;
T(2) = gas.T_ph(P(2),h(2));
s(2) = gas.s_ph(P(2),h(2));
ef(2) = ef_fun(h(2),s(2));

% come back to sp 3
P(3) = P(2);

% come back to sp 4
P(4) = P(3);

% state point 5
P(5) = P(4);
T(5) = 550; %C
h(5) = gas.h_pT(P(5),T(5));
s(5) = gas.s_pT(P(5),T(5));
ef(5) = ef_fun(h(5),s(5));


% expand in turbine for sp 6
P(6) = P(1); % kPa
s_s(6) = s(5);
h_s(6) = gas.h_ps(P(6),s_s(6));
h(6) = h(5) - eta_turb*(h(5) - h_s(6));
s(6) = gas.s_ph(P(6),h(6));
T(6) = gas.T_ph(P(6),h(6));
ef(6) = ef_fun(h(6),s(6));

% state point 7 (what I know for now...)
P(7) = P(6);

% state point 8 ( what I know for now...)
P(8) = P(7);


% set up system of equations:
% F = [f1, h(3), h(4), h(7), h(8)]

% #1 - Low Temperature Regenerator Effectiveness
LTR_eff = @(F) xi_ltr*(F(4)- gas.h_pT(P(8),T(2))) - (F(4) - F(5));

% #2 - High Temperature Regenerator Effectiveness
HTR_eff = @(F) xi_htr*(h(6)-gas.h_pT(P(7),gas.T_ph(P(3),F(2)))) - ...
    (h(6) - F(4));

% #3 - Low Temperature Regenerator Energy Balance
LTR_ebal = @(F) F(1)*(F(2)-h(2)) - (F(4)-F(5));

% #4 - High Temperature Regenerator Energy Balance
HTR_ebal = @(F) (F(3) - F(2)) - (h(6)-F(4));

% #5 - State Point 3 "identity" <- any ideas for a better name?
SP3_id = @(F) F(5) - (F(5)-gas.h_ps(P(3),gas.s_ph(P(8),F(5))))./eta_rc - ...
    F(2);

% overall balance equation.  Scalar output to vector input. Minimum value
% is zero.
balance = @(F) abs(LTR_eff(F))+abs(HTR_eff(F))+abs(LTR_ebal(F))+...
    abs(HTR_ebal(F))+abs(SP3_id(F));

option = 1;
% 1 = no inequality constraints and option arguements
% 2 = additional option arguments and inequality constraints

Aeq = []; % no linear equalities to impose
Beq = [];
lb = [0 h(2) h(2) h(1) h(2)]; % lower bound
ub = [1 h(6) h(5) h(6) h(6)]; % upper bound
initial_guess = [0.5 h(2) h(5) h(6)-1 h(2)];

switch option
    case 1
        A = []; % no linear inequality constraints
        b = [];
        [F,fval,exitflag] = fmincon(balance,initial_guess,A,b,Aeq,Beq,...
            lb,ub);
    case 2
        A = [0 1 -1 0 0;
            0 0 0 -1 1]; %< h(3) - h(4) <= 0, h(8)-h(7) <= 0
        b = [0;
            0];
        options = optimoptions('fmincon','FunctionTolerance',1e-6,...
            'ConstraintTolerance',1e-9,'StepTolerance',1e-15,...
            'Display','iter','Algorithm','sqp',...
            'MaxFunctionEvaluations',1500);
        [F,fval,exitflag] = fmincon(balance,initial_guess,A,b,Aeq,Beq,...
            lb,ub,[],options);
    otherwise
        error('Invalid selection for option!');        
end

fprintf('fval = %g \n',fval);
fprintf('exit flag = %d \n',exitflag);

% unpack the results
ff = F(1);
h(3) = F(2);
h(4) = F(3);
h(7) = F(4);
h(8) = F(5);

% fill in missing details on state point data
T(3) = gas.T_ph(P(3),h(3));
s(3) = gas.s_pT(P(3),T(3));
ef(3) = ef_fun(h(3),s(3));

T(4) = gas.T_ph(P(4),h(4));
s(4) = gas.s_ph(P(4),h(4));
ef(4) = ef_fun(h(4),s(4));

T(7) = gas.T_ph(P(7),h(7));
s(7) = gas.s_ph(P(7),h(7));
ef(7) = ef_fun(h(7),s(7));

T(8) = gas.T_ph(P(8),h(8));
s(8) = gas.s_ph(P(8),h(8));
ef(8) = ef_fun(h(8),s(8));

fprintf('Flow fraction f = %g \n',ff);

w_comp = ff*(h(1) - h(2));
w_rc = (1-ff)*(h(8) - h(3));
w_turb = h(5) - h(6);
w_net = w_comp + w_rc + w_turb;

q_s = h(5) - h(4);
q_r = ff*(h(1) - h(8));
q_net = q_s+q_r;

fprintf('Net specific work = %g kJ/kg \n',w_net);
fprintf('Net specific heat = %g kj/kg \n',q_net);

eta_th = w_net/q_s;
fprintf('Thermal efficiency = %5.2f percent \n',eta_th*100);

bwr = abs(w_comp + w_rc)/w_turb;
fprintf('Back work ratio = %g percent \n',bwr*100);

ef_in = ef(5) - ef(4);
ef_out = ff*(ef(8) - ef(1));

Ex_d_ltr = ff*ef(2) + ef(7) - ef(8) - ff*ef(3);
Ex_d_htr = ef(3) + ef(6) - ef(4) - ef(7);

fprintf('Specific flow exergy in: %g kJ/kg \n',ef_in);
fprintf('Specific flow exergy out: %g kJ/kg \n',ef_out);
fprintf('Specific flow exergy destroyed LTR: %g kJ/kg \n',Ex_d_ltr);
fprintf('Specific flow exergy destroyed HTR: %g kJ/kg \n',Ex_d_htr);

Flow_ex_balance = (ef_in - ef_out) - (w_net + Ex_d_ltr + Ex_d_htr);
fprintf('Specific flow exergy balance: %g kJ/kg \n',Flow_ex_balance);

fprintf('\n\nState point data: \n\n');

% display state point data neatly
SP = {'1','2','3','4','5','6','7','8'};
SP_table = table(P,T,h,s,ef,'RowName',SP);
disp(SP_table);
\end{lstlisting}
\end{fullwidth}

