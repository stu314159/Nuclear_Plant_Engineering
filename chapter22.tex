\chapter{Lecture 22 - Fuel Thermal Conductivity with MATLAB}
\label{ch:ch22}
\section{Objectives}
The objective of this lecture is:
\begin{enumerate}
\item Explain the implementation of the FRAPCON fuel thermal conductivity model in a MATLAB environment.
\end{enumerate}

\section{FRAPCON Model---restated} \index{FRAPCON, MATLAB}

The FRAPCON Model was discussed in Lecture 22, but in brief, the equation is as follows:

\begin{fullwidth}
\begin{equation*}
k_{0.95 \ \text{TD}} = \left\{A + BT + a \cdot \text{gad} + f(Bu) +  \left[1-0.9e^{-0.04 Bu} \right]g(Bu)h(T) \right\}^{-1} +  \frac{E}{T^2}e^{-\sfrac{F}{T}}
\end{equation*}
\end{fullwidth}
where the constants and functions included in this definition can be obtained from the previous lecture.

We adjust for density as follows:
\begin{equation*}
\eta(\rho) = \frac{1.0789 \sfrac{\rho}{\rho_{\text{TD}}}}{1 + 0.5\left(1 - \sfrac{\rho}{\rho_{\text{TD}}} \right)} 
\label{eq:FRAP_density2}
\end{equation*}
so that $k(\rho) = \eta(\rho)k_{0.95 \ \text{TD}}$.

\section{MATLAB implementation}
The first step will be to establish a MATLAB representation for the constants and functions that are used in the correlation:

\begin{lstlisting}[caption=Step 1.]
% range of applicability
% 300K < T < 3000 K
% Bu < 62 GWd/Mt
% Gad < 10 w.t. %
% 92% < TD < 97%

a_frapcon = 115.99; % cm-K/W
f_frapcon = @(Bu) 0.187*Bu; %cm-K/W
g_frapcon = @(Bu) 3.8*(Bu.^0.28);% cm-K/W
Q_frapcon = 6380; %K
h_frapcon = @(T) 1./(1 + 396*exp(-Q_frapcon./T)); % unitless
\end{lstlisting}

Since different types of oxide fuels are possible, we set up a selection structure to pick between UO$_{2}$ and MOX.
\begin{lstlisting}[caption=Step 1 continued.]
% set constants based on fuel type.
fuel_type = 'UO2'; % fuel type = 'UO2' or 'MOX'
switch fuel_type
    case 'UO2'
        A_frap = 4.52; %cm-K/W
        B_frap = 2.46e-2; %cm/W
        E_frap = 3.5e7; %W-K/cm
        F_frap = 16361; %K
       
    case 'MOX'
        O_M = 2.0; % set for MOX fuel - default 2.0
        x_frap = 2.0 - O_M;
        A_frap = 285*x_frap + 3.5; %cm-K/W
        B_frap = (2.86 - 7.15*x_frap)*1e-2; %cm/W
        E_frap = 1.5e7; %W-K/cm
        F_frap = 13520; %K
    otherwise
        % raise an error
        error('Unknown fuel type for FRAPCON model\n');
end
\end{lstlisting}

The next step will be to create a MATLAB representation of the FRAPCON thermal conductivity correlation.  You need to include the density correction function $\eta(\rho)$ if you want to model fuels with density other than 95 percent theoretical density.\marginnote{\textbf{Note:} Be sure to write your local functions to they can accept vector arguments.}

\begin{lstlisting}[caption=MATLAB function for the FRAPCON correlation.]
% equation 8.22a
k_0p95_frap = @(T,gad,Bu) ...
    1./(A_frap + B_frap*T+a_frapcon*gad+f_frapcon(Bu)+...
    (1 - 0.9*exp(-0.04*Bu)).*g_frapcon(Bu).*h_frapcon(T))+...
    E_frap./(T.^2).*exp(-F_frap./T);
% (be careful with syntax to allow any argument to be a vector)

% correction factor for theoretical density
eta_frap = @(den) 1.0789*den/(1+0.5*(1-den));

% combined expressions
k_frap = @(T,gad,Bu,den) k_0p95_frap(T,gad,Bu)*eta_frap(den);

\end{lstlisting}

Now we have the correlation available in our environment, it is time to apply problem specific parameters such as burn-up, density, and gadolinium concentration.

\begin{lstlisting}[caption=Apply problem-specific parameters.]
%% Given Parameters
T_fo = 400; % C
q_linear = 25; % kW/m
gad = 0.00; % mass fraction of gadolinia
Bu = 55; % GWd/MT_ihm
den = 0.92; % percent theoretical density

% Function that now includes
% problem-specific parameters.
k_frap_p = @(T) k_frap(T,gad,Bu,den);

\end{lstlisting}

Recall that the equation we need to solve is:
\begin{equation}
\int_{T(R)}^{T(0)} k(T,gad,Bu)\eta(\rho) \ dT = \frac{q^{\prime}}{4 \pi}
\label{eq:heat_final_ch22}
\end{equation}
where we need, for instance, to find the maximum center line temperature, $T(0)$, resulting from a given fuel outer temperature and linear heat rate.  This is in general a non-linear equation.  We can make use of MATLAB's powerful and reliable built-in non-linear root-finding function ``fzero'' if we re-arrange Equation \ref{eq:heat_final_ch22} as follows:

\begin{equation}
\int_{T(R)}^{T(0)} k(T,gad,Bu)\eta(\rho) \ dT - \frac{q^{\prime}}{4 \pi} = 0
\end{equation}

We do this in the listing below where, please note, we have to do some unit conversions to make sure that $q^{\prime}$ is compatible with the units of thermal conductivity.\marginnote{\textbf{Note:} The built-in function \emph{integral} will call the in-line function \emph{k\_frap\_p} with vectors of ``integration points.''  If you failed to implement \emph{k\_frap\_p} so that it could correctly accept vector arguments, \emph{integral} will fail. }

\begin{lstlisting}[caption=Set up non-linear equation for solver.]
f_fuel = @(T) integral(k_frap_p,T_fo+273.15,T) - ...
    (q_linear*1000/100)/(4*pi);
\end{lstlisting}

Lastly we will provide an initial guess for $T(0)$ and invoke ``fzero'' to get the actual fuel center line temperature.

\begin{lstlisting}[caption=Solve for fuel centerline temperature.]
T_est = 2250; 
T_max_K = fzero(f_fuel,T_est+273.15);
T_max = T_max_K - 273.15;

% report the result
fprintf('The fuel center line temperature is %5.4g degrees C.\n',T_max);
\end{lstlisting}

You should compare the result you get to the result you obtain using the graphical method described in Lecture 21.  Full code for this example is proved in the Appendices.
