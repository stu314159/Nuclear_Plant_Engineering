\chapter{Example - LOCA Analysis with MATLAB}
\label{app:LOCA_ex}
\section{Introduction}

This appendix provides the full MATLAB code to carry out an analysis of a Loss of Coolant Accident (LOCA) for a PWR containment using MATLAB.  The appendix includes two nearly identical analysis:
\begin{enumerate}
\item LOCA analysis without considering humidity in the air; and
\item LCOA analysis \emph{with} considering humidity in the air.
\end{enumerate}

The problem statement is, as close as possible, the same as given by Example 7.1 in the Todreas textbook.  The results for both analysis are very similar; the final temperature matches to within a degree F, and a couple of kPa.

\begin{fullwidth}
\section{LOCA analysis without humid air}
\begin{lstlisting}
clear
clc
close 'all'

EasyProp_path = ' '; %<- Path if EasyProp.py is in your current directory
if count(py.sys.path,EasyProp_path) == 0  % <-- see if desired directory is on path
    insert(py.sys.path,int32(0),EasyProp_path); %<-- if not; add it.
end


%% establish air and water objects

units = 'SI';
water = py.EasyProp.simpleFluid('Water',units);
air = py.EasyProp.simpleFluid('Air',units);

%% set initial temperature, pressure, and volumes
V_primary = 354; % m^3, given
V_containment = 50970; % m^3, given
V_total = V_primary + V_containment; % m^3

Po_primary = 15500; % kPa, initial primary pressure, given
To_primary = 344.75; % C, initial primary temperature, given

Po_containment = 101; % kPa, initial containment pressure, given
To_containment = 26.85; % C, initial containment temperature, given

Q_total = 0; % kJ, total net energy exchange in/out of the containment during the transient.

%% compute the mass of air and water

rho_a_0 = 1./air.v_pT(Po_containment,To_containment); % kg/m^3
mass_air = V_containment*rho_a_0; % kg

rho_w_0 = 1./water.v_pT(Po_primary,To_primary); % kg/m^3
mass_water = V_primary*rho_w_0; % kg

%% compute specific internal energy

u_air_0 = air.u_pT(Po_containment,To_containment); %kJ/kg
u_water_0 = water.u_pT(Po_primary,To_primary); % kJ/kg

%% set up system of equations to solve for the final state.

v_air_1 = V_total/mass_air; % m^3/kg 
v_water_1 = V_total/mass_water; % m^3/kg

%% System of equations
% unknowns: T_1

Energy = @(T) mass_water*(water.u_Tv(T,v_water_1) - u_water_0) + ...
    mass_air*(air.u_Tv(T,v_air_1) - u_air_0) - Q_total;

T_1 = fzero(Energy,0.5*(To_containment + To_primary));
fprintf('Final system temperature = %g degrees C\n',T_1);

%% Determine final system pressure (no IRWST, no core make-up tanks)
P1_water = water.P_vT(v_water_1,T_1);
P1_air = air.P_vT(v_air_1,T_1);

P1_total = P1_water + P1_air;

% get final water quality
x_1 = water.x_pu(P1_water,water.u_Tv(T_1,v_water_1));
fprintf('Final water quality = %g \n',x_1);

fprintf('Partial pressure of water = %g kPa \n',P1_water);
fprintf('Partial pressure of air = %g kPa \n',P1_air);
fprintf('Total pressure = %g kPa \n',P1_total);

\end{lstlisting}


\section{LOCA analysis with humid air}

\begin{lstlisting}
%% Todreas Example 7.1 Solution
clear
clc
close 'all'

EasyProp_path = ' '; %<- Path if EasyProp.py is in your current directory
if count(py.sys.path,EasyProp_path) == 0  % <-- see if desired directory is on path
    insert(py.sys.path,int32(0),EasyProp_path); %<-- if not; add it.
end


%% establish air and water objects

units = 'SI';
water = py.EasyProp.simpleFluid('Water',units);
air = py.EasyProp.simpleFluid('Air',units);
humidAir = py.EasyProp.HumidAir(units);

%% set initial temperature, pressure, and volumes
V_primary = 354; % m^3, given
V_containment = 50970; % m^3, given
V_total = V_primary + V_containment; % m^3

rel_humidity = 0.8;

Po_primary = 15500; % kPa, initial primary pressure, given
To_primary = 344.75; % C, initial primary temperature, given

Po_containment_total = 101; % kPa 


To_containment = 26.85; % C, initial containment temperature, given

Pw_1 = rel_humidity*water.Psat_T(To_containment); % kPa (Step 1, pg 331 )
% resulting Pw_1 is slightly higher than the value given in the text

Po_containment = Po_containment_total - Pw_1; % kPa, initial containment pressure
% ^^ Step 2 on page 332.

Q_total = 0; % kJ, total net energy exchange in/out of the containment during the transient.

%% compute the mass of air and water

rho_a_0 = 1./air.v_pT(Po_containment,To_containment); % kg/m^3
mass_air = V_containment*rho_a_0; % kg

% get humidity ratio for temperature, pressure and relative humidity
w = humidAir.w_PTR(Po_containment,To_containment,rel_humidity); % use humidAir object
mass_water_in_air = w*mass_air; %<-- this is from step 3.  humidAir gives a slightly higher result

rho_w_0 = 1./water.v_pT(Po_primary,To_primary); % kg/m^3
mass_water = V_primary*rho_w_0 + mass_water_in_air; % kg water in air and primary  

%% compute specific internal energy

u_air_0 = air.u_pT(Po_containment,To_containment); %kJ/kg
u_water_0 = water.u_pT(Po_primary,To_primary); % kJ/kg

%% set up system of equations to solve for the final state.

v_air_1 = V_total/mass_air; % m^3/kg 
v_water_1 = V_total/mass_water; % m^3/kg

%% System of equations
% unknowns: T_1

Energy = @(T) mass_water*(water.u_Tv(T,v_water_1) - u_water_0) + ...
    mass_air*(air.u_Tv(T,v_air_1) - u_air_0) - Q_total;

T_1 = fzero(Energy,0.5*(To_containment + To_primary));
fprintf('Final system temperature = %g degrees C\n',T_1);



%% Determine final system pressure (no IRWST, no core make-up tanks)
P1_water = water.P_vT(v_water_1,T_1);
P1_air = air.P_vT(v_air_1,T_1);

P1_total = P1_water + P1_air;

% get final water quality
x_1 = water.x_pu(P1_water,water.u_Tv(T_1,v_water_1));
fprintf('Final water quality = %g \n',x_1);

fprintf('Partial pressure of water = %g kPa \n',P1_water);
fprintf('Partial pressure of air = %g kPa \n',P1_air);
fprintf('Total pressure = %g kPa \n',P1_total);


\end{lstlisting}

\end{fullwidth}
