\chapter{Example - Supercritical Water Reactor Analysis}
\label{app:SCWR_ex}
\index{supercritical water reactor}

To run this code, create a MATLAB script containing the code in the listing below.  You should have a copy of the library EasyProp.py in the same folder.

\begin{fullwidth}
\begin{lstlisting}
%% SCWR Example Analysis
%% Prepare the environment
clear
clc
close 'all'

%% Add current directory to the Python Path
EasyProp_path = ' '; %<- Path if EasyProp.py is in your current directory
if count(py.sys.path,EasyProp_path) == 0  % <-- see if desired directory is on path
    insert(py.sys.path,int32(0),EasyProp_path); %<-- if not; add it.
end

%% Initialize Fluid Property object
fluid = 'Water';
units = 'SI';
water = py.EasyProp.EasyProp(fluid,units);

%% Initialize State Point Data Arrays

numSP = 25;
h = nan(numSP,1);
h_s = nan(numSP,1);
s = nan(numSP,1);
s_s = nan(numSP,1);
T = nan(numSP,1);
P = nan(numSP,1);
x = nan(numSP,1);
ef = nan(numSP,1);

%% Set parameter values from given data
Pmax = 24000; % kPa, reactor outlet pressure
Pm1 = 8900; % kPa, outlet pressure HP1
Pm2 = 6000; % kPa, outlet pressure HP2
Pm3 = 1000; % kPa, outlet pressure HP3
Pm4 = 200; % kPa, outlet pressure LP1
Pm5 = 4; % kPa, outlet pressure LP2

eta_t = 0.9;
eta_p = 0.88;

To_C = 16;
To = 16+273; % K, dead state temperature
Po = 101; % kPa, dead state pressure

ho = water.h_pT(Po,To_C);
so = water.s_pT(Po,To_C);

% function for specific flow exergy neglecting kinetic and potential energy
ef_fun = @(h_val,s_val) h_val - ho - To*(s_val - so);

%% Calculate Thermodynamic Properties of State Points

% state point 1 - condenser outlet, hotwell pump input
P(1) = Pm5;
x(1) = 0.0;
T(1) = water.Tsat_p(P(1));
s(1) = water.sL_p(P(1));
h(1) = water.hL_p(P(1));
ef(1)= ef_fun(h(1),s(1));

% state point 2 - hotwell pump outlet / OFWH inlet
P(2) = Pm4;
s_s(2) = s(1);
h_s(2) = water.h_ps(P(2),s_s(2));
h(2) = h(1) - (h(1) - h_s(2))./eta_p;
s(2) = water.s_ph(P(2),h(2));
T(2) = water.T_ph(P(2),h(2));
ef(2) = ef_fun(h(2),s(2));

% state point 3 - OFWH outlet
P(3) = Pm4;
x(3) = 0.0;
h(3) = water.hL_p(P(3));
T(3) = water.Tsat_p(P(2));
s(3) = water.sL_p(P(3));
ef(3) = ef_fun(h(3),s(3));

% state point 4 - CBP outlet
P(4) = Pm1;
s_s(4) = s(3);
h_s(4) = water.h_ps(P(4),s_s(4));
h(4) = h(3) - (h(3) - h_s(4))./eta_p;
s(4) = water.s_ph(P(4),h(4));
T(4) = water.T_ph(P(4),h(4));
ef(4) = ef_fun(h(4),s(4));

% state point 5 - outlet from mixing point / entry to CFWH 2
P(5) = P(4);
% remainder waits until after non-linear solver

% Skip to reactor outlet so I can get turbine exhaust conditions for CFWHs
% state point 9 - reactor outlet, HP1 inlet
P(9) = Pmax;
T(9) = 550; % C, reactor outlet temp (given) %-- note: "saturation temperature" doesn't make sense here.
h(9) = water.h_pT(P(9),T(9));
s(9) = water.s_pT(P(9),T(9));
ef(9) = ef_fun(h(9),s(9));

% state point 10 - HP1 outlet, 
P(10) = Pm1;
s_s(10) = s(9);
h_s(10) = water.h_ps(P(10),s_s(10));
h(10) = h(9) - (h(9) - h_s(10))*eta_t;
s(10) = water.s_ph(P(10),h(10));
T(10) = water.T_ph(P(10),h(10)); %<-- a little superheated (good!!)
tsat1 = water.Tsat_p(P(10)); 
ef(10) = ef_fun(h(10),s(10));

% state point 11 - HP2 outlet, HP3 inlet
P(11) = Pm2;
s_s(11) = s(10);
h_s(11) = water.h_ps(P(11),s_s(11));
h(11) = h(10) - (h(10) - h_s(11))*eta_t;
s(11) = water.s_ph(P(11),h(11));
T(11) = water.T_ph(P(11),h(11));
tsat2 = water.Tsat_p(P(11)); %<-- again, superheated
ef(11) = ef_fun(h(11),s(11));

% state point 12 - HP3 outlet, M/S inlet
P(12) = Pm3;
s_s(12) = s(11);
h_s(12) = water.h_ps(P(12),s_s(12));
h(12) = h(11) - (h(11) - h_s(12))*eta_t;
s(12) = water.s_ph(P(12),h(12));
x(12) = water.x_ph(P(12),h(12));
T(12) = water.T_ph(P(12),h(12));
ef(12) = ef_fun(h(12),s(12));

% state point 13 - M/S outlet, LP1 inlet
P(13) = P(12);
x(13) = 1.0;
h(13) = water.hV_p(P(13));
s(13) = water.sV_p(P(13));
T(13) = water.T_ph(P(13),h(13));
ef(13) = ef_fun(h(13),s(13));


% state point 14 - LP1 outlet , MS2 inlet
P(14) = Pm4;
s_s(14) = s(13);
h_s(14) = water.h_ps(P(14),s_s(14));
h(14) = h(13) - (h(13) - h_s(14))*eta_t;
s(14) = water.s_ph(P(14),h(14));
T(14) = water.T_ph(P(14),h(14));
x(14) = water.x_ph(P(14),h(14));
ef(14) = ef_fun(h(14),s(14));

% state point 15 - MS2 outlet, LP2 inlet
P(15) = P(14);
x(15) = 1.0;
s(15) = water.sV_p(P(15));
h(15) = water.hV_p(P(15));
T(15) = water.T_ph(P(15),h(15));
ef(15) = ef_fun(h(15),s(15));

% state point 16 - LP2 outlet, condenser inlet
P(16) = Pm5;
s_s(16) = s(15);
h_s(16) = water.h_ps(P(16),s_s(16));
h(16) = h(15) - (h(15) - h_s(16))*eta_t;
s(16) = water.s_ph(P(16),h(16));
x(16) = water.x_ph(P(16),h(16));
T(16) = water.T_ph(P(16),h(16));
ef(16) = ef_fun(h(16),s(16));

% state point 17 - MS1 liquid drain
P(17) = P(12);
x(17) = 0.0;
h(17) = water.hL_p(P(17));
s(17) = water.sL_p(P(17));
T(17) = water.T_ph(P(17),h(17));
ef(17) = ef_fun(h(17),s(17));

% state point 18 - Trap 2 outlet
P(18) = P(2);
h(18) = h(17); % isenthalpic
s(18) = water.s_ph(P(18),h(18));
T(18) = water.T_ph(P(18),h(18));
ef(18) = ef_fun(h(18),s(18));

% state point 19 - M/S #2 liquid drain
P(19) = P(14);
x(19) = 0.0;
h(19) = water.hL_p(P(19));
s(19) = water.sL_p(P(19));
T(19) = water.T_ph(P(19),h(19));
ef(19) = ef_fun(h(19),s(19));

% state point 20 - CFWH #1 drain / trap 1 inlet
P(20) = P(10);
T(20) = water.Tsat_p(P(20));
h(20) = water.hL_p(P(20));
s(20) = water.sL_p(P(20));
ef(20) = ef_fun(h(20),s(20));

% state point 21 - trap 1 outlet
P(21) = P(11);
h(21) = h(20); % isenthalpic
s(21) = water.s_ph(P(21),h(21));
T(21) = water.T_ph(P(21),h(21));
ef(21) = ef_fun(h(21),s(21));

% state point 22 - CFWH #2 drain, DBP suction
P(22) = P(11);
h(22) = water.hL_p(P(22));
s(22) = water.s_ph(P(22),h(22));
T(22) = water.T_ph(P(22),h(22));
ef(22) = ef_fun(h(22),s(22));

% state point 23 - dbp discharge
P(23) = P(4);
s_s(23) = s(22);
h_s(23) = water.h_ps(P(23),s_s(23));
h(23) = h(22) - (h(22) - h_s(23))./eta_p;
s(23) = water.s_ph(P(23),h(23));
T(23) = water.T_ph(P(23),h(23));
ef(23) = ef_fun(h(23),s(23));

% state point 24 - condenser cooling water inlet
P(24) = Po;
T(24) = To_C;
h(24) = water.h_pT(P(24),T(24));
s(24) = water.s_pT(P(24),T(24));
ef(24) = ef_fun(h(24),s(24));

% state point 25 - condenser cooling water outlet
P(25) = Po;
T(25) = 27;
h(25) = water.h_pT(P(25),T(25));
s(25) = water.s_pT(P(25),T(25));
ef(25) = ef_fun(h(25),s(25));

% state point 6 - CFWH 2 outlet / main feedwater pump inlet
P(6) = P(5);
T(6) = 230; % C, given.
h(6) = water.h_pT(P(6),T(6));
s(6) = water.s_pT(P(6),T(6));
ef(6) = ef_fun(h(6),s(6));

% state point 7 - MFWP outlet / CFWH 1 inlet
P(7) = Pmax;
s_s(7) = s(6);
h_s(7) = water.h_ps(P(7),s_s(7));
h(7) = h(6) - (h(6) - h_s(7))./eta_p;
s(7) = water.s_ph(P(7),h(7));
ef(7) = ef_fun(h(7),s(7));
T(7) = water.T_ph(P(7),h(7));

% state point 8 - CFWH1 outlet Rx inlet
P(8) = P(7);
T(8) = water.Tsat_p(Pm1); 
h(8) = water.h_pT(P(8),T(8));
s(8) = water.s_pT(P(8),T(8));
ef(8) = ef_fun(h(8),s(8));


%% Energy Balances and unknown calculation

% unknowns: f = [f1,f2,f3,h5]
OFWH_hb = @(f) (1-f(1))*(1-f(2))*x(12)*x(14)*(1-f(3))*h(2) + ...
    (1-f(1))*(1-f(2))*x(12)*x(14)*f(3)*h(15) + ...
    (1-f(1))*(1-f(2))*x(12)*(1-x(14))*h(19) + ...
    (1-f(1))*(1-f(2))*(1-x(12))*h(18) - ...
    (1-(f(1)+(1-f(1))*f(2)))*h(3);

MP1_hb = @(f) (1-(f(1)+(1-f(1))*f(2)))*h(4) + ...
    (f(1)+(1-f(1))*f(2))*h(23) - ...
    f(4); %<-- remember, f(4) = h(5)

CFWH2_hb = @(f) f(4)+ (1-f(1))*f(2)*h(11) + f(1)*h(21) - ...
    (f(1)+(1-f(1))*f(2))*h(22) - h(6);

CFWH1_hb = @(f) f(1)*h(10) + h(7) - f(1)*h(20) - h(8);
    
 balance = @(f) abs(OFWH_hb(f))+abs(MP1_hb(f))+abs(CFWH2_hb(f))+...
     abs(CFWH1_hb(f));
 
% % provide inputs to fmincon to solve non-linear equations
x0 = [0.5 0.5 0.5 h(4)];
A = []; % no linear inequality constraints
b = [];
Aeq = []; % no linear equality constraints
beq = [];
lb = [0 0 0 h(1)]; % all values must be non-negative
ub = [1 1 1 h(6)]; % flow fraction limited to 1
nlcon = [];
options = optimoptions('fmincon','Display','iter','Algorithm','sqp',...
    'StepTolerance',1e-14,'MaxFunctionEvaluations',1000);

[f,fval,exitflag] = fmincon(balance,x0,A,b,Aeq,beq,lb,ub,nlcon,options);
% 
% % unpack and report the results
fprintf('balance function value at convergence: %g kJ/kg \n',fval);
fprintf('Flow fraction to CFWH1: %g \n',f(1));
fprintf('Flow fraction to CFWH2: %g \n',f(2));
fprintf('Flow fraction to OFWH: %g \n',f(3));
h(5) = f(4);
fprintf('Enthalpy of flow into CFWH2: %g kJ/kg \n',h(5));

% finish state point 5:
T(5) = water.T_ph(P(5),h(5));
s(5) = water.s_ph(P(5),h(5));
ef(5) = ef_fun(h(5),s(5));

% display state point data neatly
SP = {'1','2','3','4','5','6','7','8','9','10','11','12',...
    '13','14','15','16','17','18','19','20','21','22','23','24','25'};
SP_table = table(P,T,h,s,x,ef,'RowName',SP);
disp(SP_table);

%% First-law Analysis results

% pumps:
w_hwp = (h(1)-h(2))*(1-f(1))*(1-f(2))*x(12)*x(14)*(1-f(3));
fprintf('Specific work Hotwell Pump: %g kJ/kg \n',w_hwp);

w_cbp = (h(3)-h(4))*(1-(f(1)+(1-f(1))*f(2)));
fprintf('Specific work Condensate Booster Pump: %g kJ/kg \n',w_cbp);

w_dbp = (h(22)-h(23))*(f(1)+(1-f(1))*f(2));
fprintf('Specific work Drain Booster Pump: %g kJ/kg \n',w_dbp);

w_mfp = h(6) - h(7);
fprintf('Specific work main feed pump: %g kJ/kg \n',w_mfp);

w_p = w_hwp + w_cbp + w_dbp + w_mfp;

% turbines:
w_hp1 = h(9)-h(10);
fprintf('Specific work HP1: %g kJ/kg \n',w_hp1);

w_hp2 = (h(10)-h(11))*(1-f(1));
fprintf('Specific work HP2: %g kJ/kg \n',w_hp2);

w_hp3 = (h(11)-h(12))*(1-f(1))*(1-f(2));
fprintf('Specific work HP3: %g kJ/kg \n',w_hp3);

w_lp1 = (h(13)-h(14))*(1-f(1))*(1-f(2))*x(12);
fprintf('Specific work LP1: %g kJ/kg \n',w_lp1);

w_lp2 = (h(15)-h(16))*(1-f(1))*(1-f(2))*x(12)*x(14)*(1-f(3));
fprintf('Specific work LP2: %g kJ/kg \n',w_lp2);

w_t = w_hp1 + w_hp2 + w_hp3 + w_lp1 + w_lp2;

w_net = w_p + w_t;

fprintf('Net specific work: %g kJ/kg \n',w_net);

% heat supplied
q_s = h(9) - h(8);

% heat rejected
q_r = (h(1)-h(16))*(1-f(1))*(1-f(2))*x(12)*x(14)*(1-f(3));

q_net = q_s+q_r;
fprintf('Net specific heat: %g kJ/kg \n',q_net);

eta_th = w_net/q_s;
fprintf('Thermal efficiency: %5.2f percent \n',eta_th*100);


\end{lstlisting}
\end{fullwidth}


